\documentclass[finnish]{article}

% --- General packages ---

\usepackage[utf8]{inputenc}
\usepackage{lmodern}
\usepackage{microtype}
\usepackage{amsfonts,amsmath,amssymb,amsthm,booktabs,color,enumitem,graphicx}
\usepackage[pdftex,hidelinks]{hyperref}
\usepackage[finnish]{babel}

% babelbib for non-english bibliography using bibtex
\usepackage[fixlanguage]{babelbib}
\selectbiblanguage{finnish}

% add bibliography to the table of contents
\usepackage[nottoc,numbib]{tocbibind}
% tocbibind renames the bibliography, use the following to change it back
\settocbibname{Lähteet}

\title{Opiskelijayhteisön vaikutus ryhmätyöskentelyyn liittyviin asenteisiin}
\author{Kenny Heinonen}
\date{\today}
\newpage

\begin{document}

% --- Front matter ---

\maketitle        % title page
\newpage
%\makeabstract     % abstract page

\tableofcontents  % table of contents
\newpage          % clear page after the table of contents


% --- Main matter ---

\section{Johdanto}

% Write some science here.

% Esimerkkilause ja lähdeviite~\cite{esimerkki}.

Tämä työ on referaatti William M. Waiten, Michele H. Jacksonin, Amer Diwanin ja Paul M. Leonardin kirjoittamasta artikkelista \emph{"Student culture vs group work in computer science"} \cite{Waite:2004:SCV:1028174.971308}.
Artikkelissa em. mainitut henkilöt pohtivat miksi heidän yliopistoissa (University of Colorado, Stanford University) opiskelevien tietojenkäsittelytieteen opiskelijoiden ryhmätyötaidot ovat puutteelliset. Waite ym. tarkastelevat asiaa tutkimalla ryhmätyöskentelyä ja sen hyödyllisyyttä. Lisäksi tarkastellaan miten opiskelijoiden ja näiden yliopistojen muodostama opiskelijakulttuuri vaikuttaa opiskelijoiden asenteisiin ryhmätyöskentelyä kohtaan. Lopuksi esitetään ratkaisuja, joilla rohkaistaan opiskelijoita yhteistyön tekemiseen.\\

Selvittääkseen mistä huonot ryhmätyötaidot opiskelijoilla johtuivat, Waite ym. panivat alulle tutkimuksen, jossa he haastattelivat opiskelijoita ja he sijoittivat kolmelle tietojenkäsittelytieteen kurssille tarkkailijoita tekemään havaintoja opetuksesta ja opiskelijoista. Nopeasti selvisi, että lähes kaikki opiskelijat suosivat työskentelemistä yksin. Tilannetta ei parantanut ryhmätöiden lisääminen kursseilla vaan se jopa huononsi opiskelijoiden tietotaitoja, sillä ryhmätöissä jotkin opiskelijat eivät tehneet mitään ja menestyivät muiden ryhmän jäsenten vaivannäöllä.

\section{Ryhmätyöskentelyn strategiat}

Waite ym. aloittivat tutkimaan ryhmätyöskentelyä ja miettimään onko siitä edes erityistä hyötyä. He havaitsivat neljä erilaista taktiikkaa, joilla ryhmätyötä pystyi lähestymään:

Ensimmäisessä taktiikassa on ideana, että ryhmän jäsenet työskentelevät tehtävän kimpussa "peräkkäisesti"~eli toinen jatkaa siitä mihin toinen jäi. Toisessa taktiikassa tehtävä jaetaan osiin ryhmän jäsenille ja jokainen suorittaa oman osansa. Kolmannessa taktiikassa ryhmän jäsenet tekevät kaikki oman ratkaisunsa, josta valitaan sitten paras tai vaihtoehtoisesti valitaan ryhmästä taitavimmat jäsenet tekemään koko työ. Neljäs taktiikka edustaa eniten ryhmätyöskentelyn ideaa eli yhteistyötä, sillä siinä ryhmän jäsenet ovat vuorovaikutuksessa toistensa kanssa säännöllisesti tehtävän teon ajan.\\

Opiskelijat valitsivat useimmiten jonkin kolmesta ensimmäisestä taktiikasta, jotka painottavat yksin työskentelyä. Kuitenkin neljäs vaihtoehto, joka korostaa yhteistyötä on kaikista hyödyllisin. Tehtävää suorittaessa opiskelijat huomioivat usein erilaisia asioita, jotka ovat keskeisiä tehtävän suorittamisen kannalta. Ilman minkäänlaista vuorovaikutusta toisten ryhmän jäsenien kanssa nämä asiat eivät tule ryhmän yhteiseen tietoon. Myös yhteiset keskustelut tehostavat oppimista ja oikeiden päätelmien tekemistä ongelmiin liittyen verrattuna siihen, että näitä asioita pohtisi yksin. Kirjoittajat päätyivät tästä johtopäätökseen, että oikeanlaiset ryhmätyöskentelytaidot ovat hyvinkin tarpeellisia ja niihin tulisi panostaa.

\section{Opiskelijayhteisön vaikutus}

Seuraavaksi Waite ym. lähtivät tutkimaan kys. yliopistoissa vallitsevaa opiskelijakulttuuria, jonka voi käsittää myös eräänlaiseksi yhteisöksi. Tällainen yhteisö luo ja säilyttää tiettyjä kulttuurillisia asioita, joihin sisältyy mm. erilaiset käytösmallit, ajattelu- ja toimintatavat tehtävien tekoon ja työskentelyyn liittyen. Aikaisemmin mainittu Waiten ym. tekemä tutkimustyö ja opiskelijoiden haastattelu valotti opiskelijoiden käyttäytymistä ja toimintatapoja, jotka auttavat mm. ymmärtämään miksi he suosivat yksin työskentelyä ja miten nämä ajattelu-, toimintatavat ja käyttäytymismallit haittaavat mahdollista ryhmätyöskentelemistä. Näitä asioita tarkastellaan seuraavaksi.

\subsection{Yksin työskentelyn suosiminen}

Tutkimukseen osallistuneet suosivat yksin työskentelemistä. Suurin syy tälle oli, että opiskelijat näkivät tehtävät ns. tuotteina. Suoritettu tehtävä on valmis tuote, jonka laatu määrittää arvosanan. Ts. lopputulos on prosessia tärkeämpi. Muita syitä olivat, että opiskelijat eivät halunneet "raahata perässään"~osaamattomampia opiskelijoita ja he eivät halunneet joutua selvittelemään mahdollisia ryhmän jäsenten välisiä ongelmia (esim. aikataulutus).

\subsection{Tehtävän teon lykkääminen}

Suunnilleen jokainen haastateltava kertoi lykkäävänsä tehtävien tekoa, vaikkakin tälle ei löytynyt selvää perustelua. Tehtävien tekemisen viivyttäminen on haitallista yhteis- ja ryhmätyötä ajatellen, koska tällöin jää vähemmän aikaa yhteisille keskusteluille.

\subsection{Välinpitämättömyys prosessia kohtaan}

Opiskelijoita lähtökohtaisesti ei kiinnostanut tehtävän suorittamisessa mikään muu kuin lopputulos ja siitä saatava arvosana. Tämä vahvistaa sitä, että opiskelijat näkevät tehtävänannon tuotteena sen sijaan, että se olisi mahdollisuus oppia uusia asioita.

\subsection{Kilpailu}

Useimmat opiskelijat esittävät omia mielipiteitään vahvasti ja hyljeksivät kanssaopiskelijoidensa mielipiteitä. Näillä opiskelijoilla on vahva käsitys omasta paremmuudestaan. Tällaisen ajattelutavan voi nähdä seurauksena yksin työskentelemisestä: kun asioita katsoo vain omien näkemyksiensä kautta, asiat tuntuvat paljon yksiulotteisemmilta. Tällainen ajattelutapa ehkäisee keskinäistä yhteisymmärrystä ja siten yhteistyötä.

\subsection{Haluttomuus tukea muita}

Monet haastateltavat tunsivat, etteivät he saa tarpeeksi tukea muilta ihmisiltä. Toiset taas mainitsivat avoimesti kieltäytyvän antamasta minkäänlaista tukea toisille. Syynä tähän nähdään, että opiskelijat haluavat muiden oivaltavan asian itse. Lisäksi avun ja tuen antamista ehkäisee käytössä olevat plagiarismiin liittyvät menettelytavat, jotka nykyisellään asettavat rajoja yhteistyölle rangaistuksen uhalla. Kuitenkin tuen antaminen ja auttaminen on tärkeä tekijä ryhmän onnistumiselle ja siten on erittäin haitallista yhteistyön kannalta, jos tätä yritetään estää tai sitä ei tehdä.

\subsection{Innokkuuden puuttuminen}

Yhteistyö voi vaatia etenkin aluksi paljon työtä ja ilman yhteistä kiinnostusta tehtävää kohtaan ryhmätyöskentely voi käydä vaikeaksi. Monet opiskelijat kertoivat ryhmätöiden suistuneen sivuraiteille ryhmän jäsenten välisten ongelmien ja kiistojen vuoksi. Näistä ongelmista on yleensä helpompi päästä yli, jos ryhmän jäseniä aidosti kiinnostaa annettu tehtävä.

\section{Yhteistyön vaaliminen}

Parantaakseen opiskelijoiden yhteistyöhalukkuutta ja -taitoja, Waiten ym. mielestä em. opiskelijayhteisöön pitää tehdä kulttuurillisia muutoksia. Tämä onnistuu luomalla olosuhteita, jotka suosivat yhteistyön sulautumista yhteisöön. Kirjoittajat kehittivät kolme erilaista tapaa vaikuttaakseen yhteisöön.

\subsection{Vuorovaikutteinen luokkahuone}

\emph{"The conversational classroom"} eli vuorovaikutteinen luokkahuone  on strategia, jolla opiskelijat saadaan näkemään yhteistyön etuja. Tässä strategiassa professori panee aluille keskustelua kurssimateriaaliin liittyen sen sijaan, että pitäisi luennon kurssimateriaalista itse. Täten professori luovuttaa informaatiovuon "oikeudet"~opiskelijoille. Tämä pakottaa opiskelijat keskustelemaan keskenään materiaalista. Tämä kokeilu on ollut erittäin onnistunut ja perinteisen luennon sijaan se on kehittänyt opiskelijoiden oppimista ja ryhmätaitoja.

\subsection{Ryhmässä päätöksen tekeminen}

Opiskelijat eivät usein halunneet työskennellä ryhmissä epäonnistuneiden päätösten takia: ryhmä pohti ratkaisuja tiettyyn ongelmaan pääsemättä mihinkään lopputulokseen, jonka seurauksena haastateltava turhautui ja ratkaisi ongelman yksin. Päätöksentekoon liittyy vahvasti opiskelijoiden piirre puolustaa omia näkemyksiään ja lytätä muiden mielipiteitä, jolloin kriteerit, joihin päätös pitäisi perustaa, jäivät vähemmälle huomiolle.\\

Tämän ongelman ratkaisemiseksi keksittiin eräälle kurssille harjoitus, jossa opiskelijoiden piti valita kääntäjä-projektiin sopiva AST-tietorakenne (abstract syntax tree). Kaikkien opiskelijoiden piti ehdottaa sopivaa puuta projektia varten ja kaikista ehdotuksista jokainen valitsi omasta mielestään sopivimman puun. Seuraavalla tunnilla koko luokka yhdessä kehitti joukon kriteereitä AST-puun valitsemista varten. Lopuksi opiskelijat menivät verkkoon arvioimaan näiden eri kriteerien tärkeyden numeroarvioinnilla ja arvioivat ehdotetuille AST-puille kuinka hyvin ne täyttävät yksittäisen kriteerin ehdot. Näin saatiin jokaiselle AST-puulle oma pisteytys ja paras näistä valittiin opiskelijoiden käyttöön kurssin loppuajaksi. Muutoksia valittuun puuhun sai tehdä vain ja ainoastaan, jos kaikki opiskelijat suostuivat siihen ja jos ja vain jos AST-puun pisteytys nousi korkeammaksi muutoksesta.

\subsection{Tehtävän merkityksen pienentäminen}

Eri kursseilla annettujen isompien tehtävien ongelmana on, että ne vievät usein paljon aikaa ja siten on oikeutettua, että niiden suorittamisesta palkitaan hyvin. Tämä ei rohkaise opiskelijoita yhteistyöhön vaan vahvistaa tehtävän "tuotteistamista", jossa lopputulos ratkaisee.\\

Waite ym. ovat lähestyneet ongelmaa vähentämällä isompien tehtävien merkitystä arvostelun kannalta ja painottamalla, että kys. tehtävät eivät ole "tuotteita"~ja ainut perustelu antaa pisteitä tehtävästä on toimia pienenä kannustimena. Tehtävän teon viivyttelyä on ehkäisty antamalla viikottaiset deadlinet. Tehtävien pieni painoarvo rohkaisee opiskelijoita kokeilemaan heille tuntemattomia tekniikoita, koska jos opiskelija epäonnistuu, menetys ei ole suuri. Näiden kaikkien asioiden pitäisi rohkaista opiskelijoita yhteistyöhön, mutta näin ei kuitenkaan pakoteta tekemään.\\

Kirjoittajien mukaan tämä kokeilu on ollut onnistunut. Opiskelijat käyttävät edelleen yhtä paljon aikaa tehtäviin, mutta kokevat niiden teon opettavaisempina kuin ennen, kun tehtävän lopputulos ei ole etusijalla.

\section{Yhteenveto}

Tutkimus osoittaa, että opiskelijoiden yhteis- ja ryhmätyötaitoja kannattaa parantaa. Opiskelijayhteisön ja opetustoiminnan tarkastelu paljasti asioita, jotka jarruttavat yhteistyön mielekkyyttä. Kirjoittajat löysivät kolme tapaa vahvistaakseen yhteistyön harjoittamista, jotka ovat perinteisten luentojen korvaaminen avoimella keskustelulla ja sulauttamalla yhteistyötä painottavia prosesseja tehtäviin, kuten kappaleessa 4.2 nähtiin. Kolmas tapa on painottaa tehtävänantoa siihen suuntaan, että prosessi on tulosta tärkeämpi. Tulokset ovat osoittaneet selvää parannusta opiskelijoiden opintomenestyksessä ja ryhmätaidoissa.

\bibliographystyle{babplain}
\bibliography{lahteet}

\end{document}