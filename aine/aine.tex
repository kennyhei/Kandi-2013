% --- Template for thesis / report with tktltiki2 class ---

\documentclass[finnish]{../tktltiki2}

% tktltiki2 automatically loads babel, so you can simply
% give the language parameter (e.g. finnish, swedish, english, british) as
% a parameter for the class: \documentclass[finnish]{tktltiki2}.
% The information on title and abstract is generated automatically depending on
% the language, see below if you need to change any of these manually.
% 
% Class options:
% - grading                 -- Print labels for grading information on the front page.
% - disablelastpagecounter  -- Disables the automatic generation of page number information
%                              in the abstract. See also \numberofpagesinformation{} command below.
%
% The class also respects the following options of article class:
%   10pt, 11pt, 12pt, final, draft, oneside, twoside,
%   openright, openany, onecolumn, twocolumn, leqno, fleqn
%
% The default font size is 11pt. The paper size used is A4, other sizes are not supported.
%
% rubber: module pdftex

% --- General packages ---

\usepackage[utf8]{inputenc}
\usepackage{lmodern}
\usepackage{microtype}
\usepackage{amsfonts,amsmath,amssymb,amsthm,booktabs,color,enumitem,graphicx}
\usepackage[pdftex,hidelinks]{hyperref}

% Automatically set the PDF metadata fields
\makeatletter
\AtBeginDocument{\hypersetup{pdftitle = {\@title}, pdfauthor = {\@author}}}
\makeatother

% --- Language-related settings ---
%
% these should be modified according to your language

% babelbib for non-english bibliography using bibtex
\usepackage[fixlanguage]{babelbib}
\selectbiblanguage{finnish}

% add bibliography to the table of contents
\usepackage[nottoc,numbib]{tocbibind}
% tocbibind renames the bibliography, use the following to change it back
\settocbibname{Lähteet}

% --- Theorem environment definitions ---

\newtheorem{lau}{Lause}
\newtheorem{lem}[lau]{Lemma}
\newtheorem{kor}[lau]{Korollaari}

\theoremstyle{definition}
\newtheorem{maar}[lau]{Määritelmä}
\newtheorem{ong}{Ongelma}
\newtheorem{alg}[lau]{Algoritmi}
\newtheorem{esim}[lau]{Esimerkki}

\theoremstyle{remark}
\newtheorem*{huom}{Huomautus}


% --- tktltiki2 options ---
%
% The following commands define the information used to generate title and
% abstract pages. The following entries should be always specified:

\title{Esimerkkiotsikko}
\author{Kenny Heinonen}
\date{\today}
\level{Aine}
\abstract{Tiivistelmä.}

% The following can be used to specify keywords and classification of the paper:

\keywords{avainsana 1, avainsana 2, avainsana 3}
\classification{} % classification according to ACM Computing Classification System (http://www.acm.org/about/class/)
                  % This is probably mostly relevant for computer scientists

% If the automatic page number counting is not working as desired in your case,
% uncomment the following to manually set the number of pages displayed in the abstract page:
%
% \numberofpagesinformation{16 sivua + 10 sivua liitteissä}
%
% If you are not a computer scientist, you will want to uncomment the following by hand and specify
% your department, faculty and subject by hand:
%
% \faculty{Matemaattis-luonnontieteellinen}
% \department{Tietojenkäsittelytieteen laitos}
% \subject{Tietojenkäsittelytiede}
%
% If you are not from the University of Helsinki, then you will most likely want to set these also:
%
% \university{Helsingin Yliopisto}
% \universitylong{HELSINGIN YLIOPISTO --- HELSINGFORS UNIVERSITET --- UNIVERSITY OF HELSINKI} % displayed on the top of the abstract page
% \city{Helsinki}
%


\begin{document}

% --- Front matter ---

\maketitle        % title page
%\makeabstract     % abstract page

\tableofcontents  % table of contents
\newpage          % clear page after the table of contents


% --- Main matter ---

\section{Johdanto}

\section{Ryhmätyöskentelyn merkitys ketterissä menetelmissä}

% Write some science here.

Ohjelmistoalalla tarvitaan monenlaisia teknisiä taitoja, jotta etenkin suuremmissa projekteissa saadaan toteutettua kaikki kehityksen vaiheet kattavasti. Nämä ohjelmistokehityksen vaiheet voidaan jakaa karkeasti vaatimusmäärittelyyn, suunnitteluun, toteutukseen, testaukseen ja ylläpitoon~\cite{Capretz:2010:MSS:1726559.1726574}. Sen lisäksi, että eri vaiheet vaativat eri taitoja, myös yksittäinen vaihe kysyy laajaa osaamista. Tämän seurauksena tulee tarve koota joukko osaavia ihmisiä toteuttamaan yhteistyössä kaikki kehityksen vaiheet. Onkin hyvin tavanomaista, että ohjelmistoprojektit toteutetaan ryhmätyönä. Useat prosessimallit jopa sanelevat miten ryhmätyöskentely tapahtuu, jotta kehitys sujuisi luontevammin ja tuotteliaammin.\\

Ryhmätyöskentelyä voi tehdä monin tavoin. On esimerkiksi tilanteita, joissa ryhmät sijaitsevat samoissa oloissa, mutta työskentelevät silti täysin erillään toisistaan. Tämän seurauksena kommunikointi kärsii ja voi olla epäsäännöllistä. Ketterät menetelmät pohjautuvat 12 ketterän kehityksen periaatteeseen [AgileManifesto]. Ryhmätyöskentelyllä ja yhteistyöllä on ketterissä
menetelmissä suuri painoarvo. Esimerkiksi eräs periaate on, että "parhaat arkkitehtuurit, vaatimukset ja suunnitelmat syntyvät itseorganisoituvissa
tiimeissä". Tämä tarkoittaa pitkälti sitä, että tiimit työskentelevät yhdessä päättäen näistä asioista ilman, että kukaan ulkopuolinen tulee
kertomaan heille mitä tehdä. Toisin sanoen tiimiä johtaa tiimi itse, tiimillä ei ole johtajaa tai projektipäällikköä. Periaatteissa painotetaan
myös informaation välittämistä kasvokkain käytävillä keskusteluilla. Ryhmillä tulisi olla yhteiset työtilat, jolloin tämä periaate toteutuisi.
Kasvokkainen kommunikointi virtaviivaistaa tiedon kulkua ja pienentää väärinkäsitysten mahdollisuutta.
Ketterissä menetelmissä toimivan ohjelmiston tuottaminen säännöllisin väliajoin on tärkeää ja sillä halutaan pitää asiakas tyytyväisenä.
Tuotteen kehitys siis tehdään iteratiivisesti ja inkrementaalisesti. Tämä tarkoittaa sitä, että tuotetta kehitetään lyhyissä kehityssykleissä pala kerrallaan, saaden jatkuvaa palautetta asiakkaalta. Seuraavissa osioissa tarkastellaan muutamia ketteriä prosessimalleja nähdäksemme miten ryhmätyöskentely ja kommunikointi otetaan niissä huomioon.

\subsection{Extreme Programming}

\emph{Extreme Programming} (XP) on ketterä menetelmä, jonka tunnettu ohjelmistokehittäjä Kent Beck on luonut. XP on iteratiivinen ja
inkrementaalinen prosessimalli, jonka tarkoituksena on tuottaa asiakkaalle mahdollisimman paljon arvoa mahdollisimman tehokkaasti.
XP:ssä kehitys tapahtuu lyhyissä iteraatioissa, jolloin vaatimusten muutoksiin on helpompi varautua, kun tavoitteet on asetettu lähitulevaisuuteen. Yhteistyö ja kommunikointi on XP:n tärkeimpiä piirteitä. Kommunikointia tapahtuu säännöllisesti asiakkaan kanssa, jotta kehittäjät voivat saada arvokasta palautetta tekemästään työstä ja siten tehdä parannuksia. Asiakas pysyy tyytyväisenä, kun saa olla jatkuvasti osana projektia ja näkee työn edistymistä. Myös ryhmän sisäinen kommunikointi on suuressa osassa sillä hyvät ratkaisut saadaan useimmiten yhteistyön tuloksena. Kommunikoinnin ja yhteistyön merkitystä kuvastaa esimerkiksi \emph{pariohjelmointi} ("pair programming"), joka on yksi XP:n keskeisimpiä käytänteitä.
Tutustutaan pariohjelmointiin tarkemmin.

\subsubsection{Pariohjelmointi}

Pariohjelmoinnissa kaksi henkilöä ohjelmoivat pareittain, jakaen saman tietokoneen. Henkilöt eivät kuitenkaan ohjelmoi samaan aikaan, vaan heille on 
nimetty kaksi roolia: toinen on \emph{ajaja} ("the driver"), ja toinen \emph{navigoija} ("the navigator"). Ajajan tehtävänä on yksinkertaisesti 
kirjoittaa koodia. Navigoijan tehtävänä on analysoida jatkuvasti kirjoitettua koodia ja kertoa ajajalle mitä tehtäviä heidän tulee milloinkin 
toteuttaa. Näin ajaja voi keskittyä pelkästään ohjelmointiin. Sovituin väliajoin henkilöt vaihtavat rooleja. Pariohjelmointi voidaan nähdä 
ryhmätyöskentelynä - kaksi ihmistä suorittavat yhteistä tehtävää saavuttaakseen saman päämäärän.~\cite{Shore:2007:AAD:1407480}\\

Pariohjelmointi korostaa yhteistyötä ja kommunikointia. Tästä on monia hyötyjä, jotka tekevät hyvää sekä ryhmän jäsenelle, ryhmälle että 
projektille~\cite{Begel:2008:PPW:1414004.1414026}. Tarkastellaan näitä hyötyjä seuraavaksi. 

\begin{itemize}

\item {\bf Koodin laatu}

Kun kaksi henkilöä pariohjelmoivat ja ratkaisevat samaa ongelmaa, lopputulos on usein tehokkaampi verrattuna siihen, että yksi henkilö tekisi kaiken. Parit pystyvät helposti keskustelemaan keskenään siitä mitä heidän tulisi seuraavaksi tehdä ja he voivat jakaa ideoita saadakseen koodista laadukkaamman tai ratkaistakseen jonkin ongelman. Sivustakatsojana navigoija pystyy tekemään tärkeitä huomioita ja pohtimaan kuinka koodista saataisiin laadukkaampi, kun taas ajaja voi keskittyä ohjelmoimiseen. Navigoija tekee ajoittain ehdotuksia ajajalle, jolla koodista saataisiin parempaa.
Pariohjelmoinnin ansiosta tapahtuva laajamittaisempi koodin katselmointi vähentää virheiden määrää. Vaihtoehtoisesti
niitä ei edes synny, kun parit keskenään kommunikoivat miettien hyviä ratkaisuja.

Pariohjelmointi parantaa myös keskittymiskykyä. Toisen henkilön läsnäolo estää herkemmin yksilöä laiskottelemasta tai rikkomaan XP:n
vaalimia käytänteitä, kuten TDD:tä. Käytänteiden noudattaminen taas johtaa parempaan koodin laatuun ja toteutukseen.

\item {\bf Oppiminen}

Kun henkilöt "pariutuvat"~toistensa kanssa, niin usein osaaminen leviää kehittäjien kesken. Monet tykkäävät neuvoa toisiaan ja ryhmän jäsenillä
on eri taitoja ja tietämystä asioista. Esimerkiksi ryhmän jäsenet, jotka pääsevät heitä taitavampien ihmisten pareiksi voivat oppia
paljon uusia tekniikoita. Parhaimmassa tapauksessa koko ryhmän sisällä kaikki voivat oppia toisiltaan jotakin.
Oppimiseen sisältyy myös se, että ymmärrys kehitettävästä ohjelmasta parantuu. Jos henkilöt vaihtavat
pareja eri ihmisten kanssa, he pääsevät näkemään erilaisia kehityksessä olevia ohjelman osia. He myös samalla osallistuvat tämän yhden osan
kehitykseen, jolloin ymmärrys koko projektista kasvaa korkeammalle tasolle.\\

Tästä voidaan päätellä, että pariohjelmointi on suuressa roolissa XP:ssä ja ryhmän jäsenien keskinäisellä vuorovaikutuksella on suuri hyöty projektin laadun kannalta. Kaikkien näiden pariohjelmoinnin hyvien puolien yhteisenä tekijänä on se, että parit kommunikoivat toistensa kanssa. Ilman keskinäistä vuorovaikutusta ei voi odottaa, että edellä mainitut asiat toteutuisivat. Pariohjelmointi kuitenkin rohkaisee kehittäjiä puhumaan toisilleen, joten tästä ei pitäisi olla huolta~\cite{Zarb:2012:UCW:2384716.2384738}.

\end{itemize}

\subsection{Scrum}

Scrum on prosessimalli, joka painottaa tiimien yhteistyötä projektin kehityksessä~\cite{ScrumORG}. Scrum, kuten XP, on myös iteratiivinen ja inkrementaalinen
eli kehitys tapahtuu lyhyissä sykleissä, \emph{"Sprinteissä"}, rakentaen kehitettävää tuotetta tietyt toiminnallisuudet kerrallaan. Lyhyet kehityssyklit sallivat tiimien mukautuvan asiakkaalta saatavaan palautteeseen ja vaatimusten muutoksiin. Näin tuotteeseen voidaan tehdä ajoissa
muutoksia ilman suuria kuluja ja tuote "hioutuu"~yhä lähemmäs sitä mitä asiakas sen haluaa olevan.\\

Projektin alussa luodaan lista toiminnallisuuksista,
joita asiakas haluaa toteutettavan. Lista on niin sanottu \emph{tuotteen kehitysjono} ("product backlog"). Backlogin sisältämät asiat priorisoidaan siten, että tärkeimmät toiminnallisuudet ovat listan kärjessä. Yksittäinen tiimi valitsee sitten Sprinttiin kehitysjonon kärjestä sen verran
toteutettavia asioita, jonka verran he uskovat pystyvänsä toteuttamaan Sprintin loppuun mennessä. Tavoitteena on toteuttaa nämä vaatimukset niin,
että tuote on "valmis"~eli toimii näiden vaatimusten osalta moitteettomasti, vaikka tuote itsessään ei sisältäisikään vielä kaikkia niitä
toiminnallisuuksia, joita asiakas haluaa. Sprintin lopuksi on \emph{Sprintin katselmus} ("Sprint review"), jossa tiimi näyttää asiakkaalle
aikaansaannoksiaan. Näin asiakas näkee projektin edistymistä säännöllisesti ja voi samalla antaa mielipiteensä tiimin tekemästä työstä. Asiakkaan
antama palaute otetaan huomioon ja asiakkaan vaatimat muutokset kirjataan kehitysjonoon.\\

Scrumissa tiimit koostuvat \emph{monitaitoisista} ("cross-functional") jäsenistä, joilta löytyy tarpeellinen tekninen osaaminen, jota vaaditaan
tuotteen toteuttamiseksi. Tiimillä ei ole johtajaa tai projektipäällikköä, jotka määräisivät tiimin tekemisistä. Sen sijaan tiimi itse
saa päättää tavoitteet joka Sprintille ja sen miten nämä tavoitteet saavutetaan. Toisin sanoen tiimi on \emph{itseorganisoituva}
("self-organizing"). Tämä on Scrumille olennaista ja siitä näkee kuinka suuressa arvossa tiimin jäsenten välistä yhteistyötä pidetään.~\cite{ScrumHandBook}

\subsubsection{Sprintin retrospektiivi}

Sprintin lopuksi järjestetään \emph{Sprintin retrospektiivi}
("Sprint retrospective"), jossa tiimi keskustelee kuinka heidän
oma työprosessinsa sujui Sprintin aikana. Käytännössä jäsenet
keskustelevat yhdessä mikä Sprintissä sujui hyvin ja missä
asioissa pitäisi parantaa. Jäsenien antama palaute ja kritiikki
voi esimerkiksi kohdistua yksittäisen jäsenen työskentelytapoihin.
Jos mahdollisia ongelmakohtia on, tiimin tulisi yhdessä päättää
miten nämä ongelmat ratkaistaan. Retrospektiivi on tärkeä vaihe
ryhmätyöskentelyn kannalta sillä se on Scrumin pääasiallisin
mekanismi tuoda tiimin ongelmat näkyville ja ratkaista ne siten,
että ryhmän työskentely vahvistuu ja paranee.~\cite{Scrumprimer}

\subsection{Lean/FDD?}

\section{Persoonallisuuden vaikutus}

\subsection{Hyvän työkumppanin piirteet}

\subsection{Persoonallisuustyypit eri kehityksen vaiheissa}

\subsubsection{Vaatimusmäärittely}
\subsubsection{Suunnittelu}
\subsubsection{Toteutus}
\subsubsection{Testaus}
\subsubsection{Ylläpito}

\section{Ryhmätyötaitojen parantaminen}


% --- Back matter ---
%
% bibtex is used to generate the bibliography. The babplain style
% will generate numeric references (e.g. [1]) appropriate for theoretical
% computer science. If you need alphanumeric references (e.g [Tur90]), use
%
% \bibliographystyle{babalpha}
%
% instead.

\bibliographystyle{babplain}
\bibliography{../lahteet}


\end{document}