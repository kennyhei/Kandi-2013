% --- Template for thesis / report with tktltiki2 class ---

\documentclass[finnish]{../tktltiki2}

% tktltiki2 automatically loads babel, so you can simply
% give the language parameter (e.g. finnish, swedish, english, british) as
% a parameter for the class: \documentclass[finnish]{tktltiki2}.
% The information on title and abstract is generated automatically depending on
% the language, see below if you need to change any of these manually.
% 
% Class options:
% - grading                 -- Print labels for grading information on the front page.
% - disablelastpagecounter  -- Disables the automatic generation of page number information
%                              in the abstract. See also \numberofpagesinformation{} command below.
%
% The class also respects the following options of article class:
%   10pt, 11pt, 12pt, final, draft, oneside, twoside,
%   openright, openany, onecolumn, twocolumn, leqno, fleqn
%
% The default font size is 11pt. The paper size used is A4, other sizes are not supported.
%
% rubber: module pdftex

% --- General packages ---

\usepackage[utf8]{inputenc}
\usepackage{lmodern}
\usepackage{microtype}
\usepackage{amsfonts,amsmath,amssymb,amsthm,booktabs,color,enumitem,graphicx}
\usepackage[pdftex,hidelinks]{hyperref}

% Automatically set the PDF metadata fields
\makeatletter
\AtBeginDocument{\hypersetup{pdftitle = {\@title}, pdfauthor = {\@author}}}
\makeatother

% --- Language-related settings ---
%
% these should be modified according to your language

% babelbib for non-english bibliography using bibtex
\usepackage[fixlanguage]{babelbib}
\selectbiblanguage{finnish}

% add bibliography to the table of contents
\usepackage[nottoc,numbib]{tocbibind}
% tocbibind renames the bibliography, use the following to change it back
\settocbibname{Lähteet}

% --- Theorem environment definitions ---

\newtheorem{lau}{Lause}
\newtheorem{lem}[lau]{Lemma}
\newtheorem{kor}[lau]{Korollaari}

\theoremstyle{definition}
\newtheorem{maar}[lau]{Määritelmä}
\newtheorem{ong}{Ongelma}
\newtheorem{alg}[lau]{Algoritmi}
\newtheorem{esim}[lau]{Esimerkki}

\theoremstyle{remark}
\newtheorem*{huom}{Huomautus}


% --- tktltiki2 options ---
%
% The following commands define the information used to generate title and
% abstract pages. The following entries should be always specified:

\title{Ohjelmistoala ja ryhmätyöskentely}
\author{Kenny Heinonen}
\date{\today}
\level{Aine}
\abstract{Tiivistelmä.}

% The following can be used to specify keywords and classification of the paper:

\keywords{avainsana 1, avainsana 2, avainsana 3}
\classification{} % classification according to ACM Computing Classification System (http://www.acm.org/about/class/)
                  % This is probably mostly relevant for computer scientists

% If the automatic page number counting is not working as desired in your case,
% uncomment the following to manually set the number of pages displayed in the abstract page:
%
% \numberofpagesinformation{16 sivua + 10 sivua liitteissä}
%
% If you are not a computer scientist, you will want to uncomment the following by hand and specify
% your department, faculty and subject by hand:
%
% \faculty{Matemaattis-luonnontieteellinen}
% \department{Tietojenkäsittelytieteen laitos}
% \subject{Tietojenkäsittelytiede}
%
% If you are not from the University of Helsinki, then you will most likely want to set these also:
%
% \university{Helsingin Yliopisto}
% \universitylong{HELSINGIN YLIOPISTO --- HELSINGFORS UNIVERSITET --- UNIVERSITY OF HELSINKI} % displayed on the top of the abstract page
% \city{Helsinki}
%


\begin{document}

% --- Front matter ---

\maketitle        % title page
%\makeabstract     % abstract page

\tableofcontents  % table of contents
\newpage          % clear page after the table of contents


% --- Main matter ---

\section{Johdanto}

\section{Ryhmätyöskentelyn merkitys ketterissä menetelmissä}

% Write some science here.

Ohjelmistoalalla tarvitaan monenlaisia teknisiä taitoja, jotta etenkin 
suuremmissa projekteissa saadaan toteutettua kaikki kehityksen vaiheet 
kattavasti. Nämä ohjelmistokehityksen vaiheet voidaan jakaa karkeasti 
vaatimusmäärittelyyn, suunnitteluun, toteutukseen, testaukseen ja 
ylläpitoon~\cite{Capretz:2010:MSS:1726559.1726574}. Sen lisäksi, että 
eri vaiheet vaativat eri taitoja, myös yksittäinen vaihe kysyy laajaa 
osaamista. Tämän seurauksena tulee tarve koota joukko osaavia ihmisiä 
toteuttamaan yhteistyössä kaikki kehityksen vaiheet. Onkin hyvin 
tavanomaista, että ohjelmistoprojektit toteutetaan ryhmätyönä. Useat 
prosessimallit jopa sanelevat miten ryhmätyöskentely tapahtuu, jotta 
kehitys sujuisi luontevammin ja tuotteliaammin.\\

Ryhmätyöskentelyä voi tehdä monin tavoin. On esimerkiksi tilanteita, 
joissa ryhmät sijaitsevat samoissa oloissa, mutta työskentelevät silti 
täysin erillään toisistaan. Tämän seurauksena kommunikointi kärsii ja 
voi olla epäsäännöllistä. Ketterät menetelmät pohjautuvat 12 ketterän 
kehityksen periaatteeseen~\cite{AgileManifesto}. Ryhmätyöskentelyllä 
ja yhteistyöllä on ketterissä
menetelmissä suuri painoarvo. Esimerkiksi eräs periaate on, että 
"parhaat arkkitehtuurit, vaatimukset ja suunnitelmat syntyvät 
itseorganisoituvissa
tiimeissä". Tämä tarkoittaa pitkälti sitä, että tiimit työskentelevät 
yhdessä päättäen näistä asioista ilman, että kukaan ulkopuolinen tulee
kertomaan heille mitä tehdä. Toisin sanoen tiimiä johtaa tiimi itse, 
tiimillä ei ole johtajaa tai projektipäällikköä. Periaatteissa 
painotetaan
myös informaation välittämistä kasvokkain käytävillä keskusteluilla. 
Ryhmillä tulisi olla yhteiset työtilat, jolloin tämä periaate 
toteutuisi.
Kasvokkainen kommunikointi virtaviivaistaa tiedon kulkua ja pienentää 
väärinkäsitysten mahdollisuutta.
Ketterissä menetelmissä toimivan ohjelmiston tuottaminen säännöllisin 
väliajoin on tärkeää ja sillä halutaan pitää asiakas tyytyväisenä.
Tuotteen kehitys siis tehdään iteratiivisesti ja inkrementaalisesti. 
Tämä tarkoittaa sitä, että tuotetta kehitetään lyhyissä 
kehityssykleissä pala kerrallaan, saaden jatkuvaa palautetta 
asiakkaalta. Seuraavissa osioissa tarkastellaan muutamia ketteriä 
prosessimalleja nähdäksemme miten ryhmätyöskentely ja kommunikointi 
otetaan niissä huomioon.

\subsection{Extreme Programming}

\emph{Extreme Programming} (XP) on ketterä menetelmä, jonka tunnettu 
ohjelmistokehittäjä Kent Beck on luonut. XP keskittyy asiakkaan tyytyväiseen~\cite{XP.ORG}. XP:n tarkoituksena on tuottaa 
asiakkaalle mahdollisimman paljon arvoa mahdollisimman tehokkaasti ja nopeasti.
XP:ssä kehitys tapahtuu lyhyissä, 1-3 viikon iteraatioissa, jolloin vaatimusten 
muutoksiin on helpompi varautua, kun tavoitteet on asetettu 
lähitulevaisuuteen. XP painottaa ketterien menetelmien tapaan kommunikointia ja ryhmätyötä.

\subsubsection{Kommunikointi}

Kasvokkaista ja usein tapahtuvaa kommunikointia painotetaan järjestämällä tiimin jäsenet yhteiseen tilaan. Tämän lisäksi projektia varten on hankittu vähintään yksi paikan päällä oleva asiakas, joka on osana kehitystiimiä nähdäkseen projektin edistymisen. Paikan päällä olevan asiakkaan kanssa keskustellaan kehityksen jokaisesta vaiheesta ja hän päättää tuotteen vaatimuksista ja niiden priorisoinnista. Asiakkaan läsnäolo on suuri etu, sillä tiimi voi kysyä häneltä hetkessä esimerkiksi jonkin toteutettavan vaatimuksen yksityiskohdista ja asiakas vastaavasti voi antaa heti palautetta työstä. Ryhmän sisäinen kommunikointi on myös suuressa osassa sillä hyvät ratkaisut saadaan useimmiten yhteistyön tuloksena. Kommunikoinnin ja yhteistyön merkitystä kuvastaa esimerkiksi \emph{pariohjelmointi} ("pair 
programming"), joka on yksi XP:n keskeisimpiä käytänteitä.
Tutustutaan pariohjelmointiin tarkemmin.

\subsubsection{Pariohjelmointi}

Pariohjelmoinnissa kaksi henkilöä ohjelmoivat pareittain, jakaen saman 
tietokoneen~\cite{Shore:2007:AAD:1407480}. Henkilöt eivät kuitenkaan ohjelmoi samaan aikaan, vaan 
heille on 
nimetty kaksi roolia: toinen on \emph{ajaja} ("the driver"), ja toinen 
\emph{navigoija} ("the navigator"). Ajajan tehtävänä on 
yksinkertaisesti 
kirjoittaa koodia. Navigoijan tehtävänä on analysoida jatkuvasti 
kirjoitettua koodia ja kertoa ajajalle mitä tehtäviä heidän tulee 
milloinkin 
toteuttaa. Näin ajaja voi keskittyä pelkästään ohjelmointiin. Sovituin 
väliajoin henkilöt vaihtavat rooleja. Pariohjelmointi voidaan nähdä 
ryhmätyöskentelynä --- kaksi ihmistä suorittavat yhteistä tehtävää 
saavuttaakseen saman päämäärän.\\

Pariohjelmointi korostaa yhteistyötä ja kommunikointia. Tästä on monia 
hyötyjä, jotka tekevät hyvää sekä ryhmän jäsenille, ryhmälle että 
projektille~\cite{Begel:2008:PPW:1414004.1414026}. Tarkastellaan näitä 
hyötyjä seuraavaksi. 

\begin{itemize}

\item {\bf Koodin laatu}

Kun kaksi henkilöä pariohjelmoivat ja ratkaisevat samaa ongelmaa, 
lopputulos on usein tehokkaampi verrattuna siihen, että yksi henkilö 
tekisi kaiken. Parit pystyvät helposti keskustelemaan keskenään siitä 
mitä heidän tulisi seuraavaksi tehdä ja he voivat jakaa ideoita 
saadakseen koodista laadukkaamman tai ratkaistakseen jonkin ongelman. 
Sivustakatsojana navigoija pystyy tekemään tärkeitä huomioita ja 
pohtimaan kuinka koodista saataisiin laadukkaampi, kun taas ajaja voi 
keskittyä ohjelmoimiseen. Navigoija tekee ajoittain ehdotuksia 
ajajalle, jolla koodista saataisiin parempaa.
Pariohjelmoinnin ansiosta tapahtuva laajamittaisempi koodin 
katselmointi vähentää virheiden määrää. Vaihtoehtoisesti
niitä ei edes synny, kun parit keskenään kommunikoivat miettien hyviä 
ratkaisuja.

Pariohjelmointi parantaa myös keskittymiskykyä. Toisen henkilön 
läsnäolo estää herkemmin yksilöä laiskottelemasta tai rikkomaan XP:n
vaalimia käytänteitä, kuten TDD:tä. Käytänteiden noudattaminen taas 
johtaa parempaan koodin laatuun ja toteutukseen.

\item {\bf Oppiminen}

Kun henkilöt "pariutuvat"~toistensa kanssa, niin osaaminen 
leviää kehittäjien kesken. Monet tykkäävät neuvoa toisiaan ja ryhmän 
jäsenillä
on eri taitoja ja tietämystä asioista. Esimerkiksi ryhmän jäsenet, 
jotka pääsevät heitä taitavampien ihmisten pareiksi voivat oppia
paljon uusia tekniikoita. Parhaimmassa tapauksessa koko ryhmän sisällä 
kaikki voivat oppia toisiltaan jotakin.
Oppimiseen sisältyy myös se, että ymmärrys kehitettävästä ohjelmasta 
parantuu. Jos henkilöt vaihtavat
pareja eri ihmisten kanssa, he pääsevät näkemään erilaisia 
kehityksessä olevia ohjelman osia. He myös samalla osallistuvat tämän 
yhden osan
kehitykseen, jolloin ymmärrys koko projektista kasvaa korkeammalle 
tasolle.\\

Tästä voidaan päätellä, että pariohjelmointi on suuressa roolissa 
XP:ssä ja ryhmän jäsenien keskinäisellä vuorovaikutuksella on suuri 
hyöty projektin laadun kannalta. Kaikkien näiden pariohjelmoinnin 
hyvien puolien yhteisenä tekijänä on se, että parit kommunikoivat 
toistensa kanssa. Ilman keskinäistä vuorovaikutusta ei voi odottaa, 
että edellä mainitut asiat toteutuisivat. Pariohjelmointi kuitenkin 
rohkaisee kehittäjiä puhumaan toisilleen, joten tästä ei pitäisi olla 
huolta~\cite{Zarb:2012:UCW:2384716.2384738}.

\end{itemize}

\subsubsection{Julkaisusuunnittelun kokous}

Kun asiakas on tehnyt listan vaatimuksia, joita kehitettävän järjestelmän pitää sisältää, aloitetaan XP:n prosessi \emph{julkaisusuunnittelun kokouksella} ("Release Planning Meeting"),
jossa on tarkoituksena tehdä \emph{julkaisusuunnitelma} ("Release Plan"). XP:ssä on tapana iteraatioiden lopuksi julkaista uusin, toimiva versio asiakkaiden käyttöön. Julkaisusuunnitelma määrittelee mitkä vaatimukset toteutetaan missäkin iteraatiossa ja siten ovat valmiina iteraation lopuksi tehtävässä julkaisussa. Näille vaatimuksille määritellään myös päivämäärät.\\

Kokouksessa kehitystiimin tehtävänä on estimoida kuinka paljon yksittäinen vaatimus vie työaikaa. Estimointien perusteella asiakas
tekee päätöksen vaatimusten prioriteeteista. Tämän jälkeen tiimin jäsenet
yhdessä asiakkaan kanssa sijoittavat vaatimukset tietyille iteraatioille toteutettaviksi.

\subsubsection{Iteraatiosuunnittelun kokous}

Ennen kuin varsinainen iteraatio alkaa, jossa itse tuotteen tiettyjen toiminnallisuuksien
kehitys tapahtuu, järjestetään \emph{iteraatiosuunnittelun kokous} ("Iteration Planning Meeting"), jossa tehdään \emph{iteraatiosuunnitelma} ("Iteration Plan"). Iteraatiosuunnittelun tavoitteena on päättää mitä iteraation aikana tehdään, joka kirjataan iteraatiosuunnitelmaan.\\

Suunnittelun alussa asiakas valitsee julkaisusuunnitelmasta korkeimman prioriteetin omaavat vaatimukset alkavaan iteraatioon. Kehittäjät pilkkovat nämä vaatimukset pieniksi, toteutettaviksi tehtäviksi. Tämän jälkeen kehittäjät päättävät ketkä toteuttavat minkäkin tehtävän ja he estimoivat tehtäviin kuluvan työmäärän. Lopuksi vaatimukset, tehtävät ja niiden estimaatit kirjataan iteraatiosuunnitelmaan.

\subsubsection{Iteraatio}

Iteraatiosuunnittelun ja onnistuneen suunnitelman tekemisen jälkeen voidaan aloittaa itse iteraatio. Iteraatio kestää 1-3 viikkoa, jonka aikana iteraatiosuunnitelmaan kirjatut vaatimukset ja niihin liittyvät tehtävät pitää toteuttaa. Kehitystyö tapahtuu pariohjelmoiden, jonka
merkitystä ryhmätyöskentelyyn tarkasteltiin osiossa 2.1.2. Iteraatioon liittyy lisäksi päivittäinen palaveri, jossa ryhmän jäsenet raportoivat toisilleen mitä saivat viime palaverin jälkeen aikaan, mitä aikovat saada aikaan ennen seuraavaa palaveria ja onko heidän etenemisessään ongelmia. Näin tiimin jäsenet ovat jatkuvasti tietoisia
muiden jäsenten sekä koko tiimin tilanteesta. Kuten edellä on mainittu, iteraation aikana on vähintään yksi asiakas läsnä joka päivä, jonka kanssa voi puhua ongelmatilanteissa toteutettavien vaatimusten yksityiskohdista ja neuvotella kehityksestä.\\

Iteraation lopuksi suoritetaan \emph{hyväksymätestaus} ("Acceptance Testing"), jossa testataan iteraation aikana toteutettuja vaatimuksia ja varmistutaan siitä, että ne toimivat oikein. Asiakas on määritellyt tarkemmin mitä skenaarioita testien tulisi testata. Jos yksittäinen vaatimus läpäisee kaikki siihen kohdistuvat testit, on vaatimus toteutettu valmiiksi. Asiakas itse tarkistaa testitulokset ja päättää niiden perusteella onko viimeisin versio tuotteesta julkaisukelpoinen. Jos testauksen aikana löytyy ohjelmointivirheitä, ne kirjataan ylös ja korjataan seuraavan iteraation aikana.\\

Ketteränä menetelmänä XP on iteratiivinen, joten kun yksi iteraatio on loppunut, aloitetaan edellä mainittu prosessi uudestaan alkaen julkaisusuunnittelun kokouksella. Näin XP:n prosessia jatketaan, kunnes tuote on valmis. XP painottaa hyvin paljon asiakkaan kanssa kommunikointia ja läsnäoloa, jotta kehityksessä ei päädytä sivuraiteille. Asiakas on osana ryhmää ja hänen kanssaan tapahtuva kommunikointi ja yhteistyö vaikuttaa suuresti projektin onnistumiseen yhdessä ryhmän sisäisen kommunikoinnin kanssa.

\subsection{Scrum}

Scrum on prosessimalli, joka painottaa tiimien yhteistyötä projektin 
kehityksessä~\cite{ScrumORG}. Scrum, kuten XP, on myös iteratiivinen 
ja inkrementaalinen menetelmä.
Kehitys tapahtuu lyhyissä 1-4 viikon sykleissä, 
\emph{"Sprinteissä"}, joissa toteutetaan kehitettävää tuotetta tietyt 
toiminnallisuudet kerrallaan. Lyhyet kehityssyklit sallivat tiimien 
mukautuvan asiakkaalta saatavaan palautteeseen ja vaatimusten 
muutoksiin. Näin tuotteeseen voidaan tehdä ajoissa
muutoksia ilman suuria kuluja ja tuote "hioutuu"~yhä lähemmäs sitä 
mitä asiakas sen haluaa olevan. Scrum tekee projektin kehityksestä
monivaiheisen. Seuraavissa osioissa kerrotaan millainen Scrum-tiimi
on ja mistä vaiheista projektin elinkaari koostuu.

\subsubsection{Tiimi}

Scrumissa tiimit koostuvat \emph{monitaitoisista} ("cross-functional") 
jäsenistä, joilta löytyy tarpeellinen tekninen osaaminen, jota 
vaaditaan
tuotteen toteuttamiseksi. Tiimillä ei ole johtajaa tai 
projektipäällikköä, jotka määräisivät tiimin tekemisistä. Sen sijaan 
tiimi itse
saa päättää tavoitteet joka Sprintille ja sen miten nämä tavoitteet 
saavutetaan. Toisin sanoen tiimi on \emph{itseorganisoituva}
("self-organizing"). Tämä on Scrumille olennaista ja siitä näkee 
kuinka suuressa arvossa tiimin jäsenten välistä yhteistyötä 
pidetään~\cite{ScrumHandBook}.

\subsubsection{Scrumin aloitus}

Scrumin ensimmäinen askel on luoda visio tuotteen vaatimista
ei-toiminnallisista ja toiminnallisista vaatimuksista. Vaatimukset
kootaan listaksi, jota kutsutaan \emph{tuotteen kehitysjonoksi} ("product backlog")~\cite{ScrumFinnishGuide}. Kehitysjono on priorisoitu siten, että tärkeimmät vaatimukset ovat kehitysjonon kärjessä. Kehitysjono on jatkuvan muutoksen alaisena: uusia vaatimuksia
voi tulla lisää asiakkaan toimesta, tarpeettomia vaatimuksia karsitaan ja olemassaolevia muokataan tai tarkennetaan. Edellä mainitut tehtävät
ovat niin kutsutun \emph{tuoteomistajan} ("product owner") vastuulla,
joka on yksittäinen henkilö ja pitää huolen tuotteen arvon ja kehitystiimin työn arvon maksimoimisesta. Tuoteomistaja voi kuitenkin pyytää kehitystiimiä auttamaan edellä mainituissa tehtävissä.

\subsubsection{Sprintin suunnittelupalaveri}

Jokaisen Sprintin aluksi järjestetään kokous, jota kutsutaan \emph{Sprintin suunnittelupalaveriksi} ("Sprint planning meeting")~\cite{ScrumHandBook}. Palaveri on kaksiosainen.
Ensimmäisessä osassa tuoteomistaja ja kehitystiimi neuvottelevat
siitä mitkä vaatimukset tiimin tulisi toteuttaa alkavassa
Sprintissä. Päätösvastuu on kuitenkin tiimin jäsenillä.\\

Toisessa osassa tiimi valitsee toteutettavat vaatimukset, jotka
he sitoutuvat tekemään Sprintin loppuun mennessä. Vaatimukset
valitaan aina kehitysjonon kärjestä ylhäältä alas järjestyksessä.
Kun vaatimukset on valittu, ne hajotetaan pienemmiksi, teknisiksi
tehtäviksi. Nämä tehtävät kirjataan \emph{Sprintin tehtävälistaan} ("Sprint backlog") aika-arvioineen, mitä tiimi käyttää hyödykseen Sprintin ajan. Tiimi saa siis itse valita toteutettavat vaatimukset alkavalle Sprintille ja suunnitella miten ne toteutetaan.

\subsubsection{Scrumin päiväpalaveri}

Kun Sprintti alkaa, sen aikana harjoitetaan erästä Scrumin käytäntöä:
\emph{päiväpalaveria} ("Daily Scrum"). Tämä on lyhyt, noin 15 minuuttia kestävä kokoontuminen joka päivä samaan aikaan, johon
kaikki tiimin jäsenet osallistuvat. Jokainen tiimin jäsen saa
mahdollisuuden raportoida kolme asiaa muille tiimin jäsenille:
mitä henkilö on saanut aikaan viime tapaamisen jälkeen, mitä henkilö
aikoo saada aikaan ennen seuraavaa tapaamista ja onko mitään
esteitä työn etenemiselle? Näin tiimin jäsenet säännöllisesti tietävät
kuinka muiden työ edistyy ja palaverin jälkeen mahdollisia raportoituja
ongelmia voidaan yhdessä ratkoa. Huomionarvoista on, että on yleisesti
suositeltua, että päiväpalaveriin ei osallistu esimiehiä tai muita
ulkopuolisia henkilöitä~\cite{ScrumHandBook}. Jos näin käy, on riski,
että henkilöt tuntevat olevansa tarkkailun alaisena ja tämä voi
tuottaa heille paineita oman edistymisen tai ongelmien raportoinnista.

\subsubsection{Sprintin katselmointi}

Sprintin lopussa järjestetään \emph{Sprintin katselmointi} ("Sprint review"), jossa tiimi esittelee Sprintin aikana toteuttamiaan
vaatimuksia tuoteomistajalle ja asiakkaalle. Asiakas ja tuoteomistaja
tietävät näin miten projekti on edistynyt ja tiimi vastaavasti saa
arvokasta palautetta kyseisiltä henkilöiltä. Tämä on motivoivaa
kaikille osapuolille. Annettu palaute kirjataan mahdollisina muutoksina
vaatimuksiin tai uusina vaatimuksina tuotteen kehitysjonoon.

\subsubsection{Sprintin retrospektiivi}

Viimeisenä asiana Sprintin lopuksi järjestetään \emph{Sprintin retrospektiivi}
("Sprint retrospective"), jossa tiimi keskustelee kuinka heidän
oma työprosessinsa sujui Sprintin aikana~\cite{Scrumprimer}. Käytännössä jäsenet
keskustelevat yhdessä mikä Sprintissä sujui hyvin ja missä
asioissa pitäisi parantaa. Jäsenien antama palaute ja kritiikki
voi esimerkiksi kohdistua yksittäisen jäsenen työskentelytapoihin.
Jos mahdollisia ongelmakohtia on, tiimin tulisi yhdessä päättää
miten nämä ongelmat ratkaistaan. Retrospektiivi on tärkeä vaihe
ryhmätyöskentelyn kannalta sillä se on Scrumin pääasiallisin
mekanismi tuoda tiimin ongelmat näkyville ja ratkaista ne siten,
että ryhmän työskentely vahvistuu ja paranee.\\

Kun nämä kaikki edellä mainitut vaiheet on käyty läpi, alkaa uusi
Sprintti. Käytännössä siis aloitetaan uusi sykli aloittaen
Sprintin suunnittelupalaverista. Tätä periaatteessa jatketaan
niin kauan, kunnes tuote on valmis eli asiakkaan kaikki vaatimukset
on toteutettu. Kuten monista asioista kävi ilmi, kehitystiimin jäsenillä on paljon valtaa sen suhteen mitä he tekevät ja miten he
sen tekevät. Suunnittelutyö ja toteutus on asia, jonka tiimin
jäsenten pitää yhteisymmärryksessä päättää kommunikoiden toistensa kanssa.

\section{Persoonallisuuden vaikutus}

Kehitystiimit koostuvat erilaisista ihmisistä ja useissa tutkimuksissa on huomattu, että tietyt
luonteenpiirteet ja persoonallisuustyypit vaikuttavat positiivisesti tiimin suorituskykyyn sekä
projektin onnistumiseen~\cite{Acuna:2008:ESP:1414004.1414056,Gorla:2004:WWB:990680.990684,Capretz:2003:PTS:766407.766410,Capretz:2010:MSS:1726559.1726574}. Sen lisäksi, että tiimien olisi hyvä koostua monitaitoisista
jäsenistä, moninaiset persoonallisuustyypit tiimin sisällä ovat hyväksi
projektille. Esimerkiksi suurempia ratkaisuja mietittäessä harvoin hyväksytään
ensimmäinen ehdotus, joka esitetään, vaan puntaroidaan monen ehdotuksen
välillä arvioiden niiden hyviä ja huonoja puolia. Myös yksilön oma
tekninen osaaminen vaikuttaa ratkaisun miettimiseen. Eri tavalla ajattelevat ihmiset kykenevät tuomaan joukon
eri näkökulmia tuotetta kehitettäessä, joka johtaa parempiin ratkaisuihin.\\

Edellä mainitun lisäksi henkilön persoonallisuus vaikuttaa myös
siihen
kuinka mieluista kyseisen henkilön kanssa on työskennellä,
miten henkilö lähestyy annettua tehtävää ja minkälaisiin tehtäviin
hän soveltuu parhaiten~\cite{Begel:2008:PPW:1414004.1414026,Capretz:2010:MSS:1726559.1726574}.
Seuraavissa osioissa tarkastellaan millaisia piirteitä hyvällä
ohjelmistokehittäjällä on ja mitkä erilaiset persoonallisuustyypit
sopivat parhaiten projektien eri kehitysvaiheisiin.

\subsection{Hyvän ohjelmistokehittäjän piirteet}

Hyvälle ohjelmistokehittäjälle on aina tilaa ohjelmistoalalla.
Siinä missä tekninen osaaminen ja hyvä tietämys asioista on
tärkeää, myös ihmistaidot ja kommunikointi on alkanut keräämään
huomiota alalla~\cite{Hall:2007:CNT:1235000.1235043}. Esimerkiksi
pariohjelmoinnissa työskentely vaatii kommunikointia ja sitä, että
tulee toisten ihmisten kanssa toimeen.
Tutkimuksissa
on haastateltu ohjelmistoalan ammattilaisia ja kysytty heiltä mitkä
piirteet ja asiat tekevät hyvän ohjelmistokehittäjän~\cite{Acuna:2008:ESP:1414004.1414056,Begel:2008:PPW:1414004.1414026,Hall:2007:CNT:1235000.1235043}. Listataan haastattelujen perusteella tärkeimpiä
piirteitä.

\begin{itemize}

\item {\bf Joustava \& mukautuva}

Joustava henkilö on avaramielinen. Henkilö kuuntelee mielellään muiden
ideoita ja kykenee katsomaan asioita eri näkökulmista sen sijaan, että
puolustelisi vain omaa kantaansa. Lisäksi hän on halukas yhteistyöhön
ja pystyy mukautumaan eri työtapoihin.

\item {\bf Hyvät kommunikointitaidot}

Ohjelmiston kehittäminen vaatii paljon ryhmätyöskentelyä ja
kommunikointia tiimin jäsenten kesken. Hyvä kommunikoija kykenee
kuuntelemaan muiden ideoita, ilmaisee omat mielipiteensä selkeästi ja uskaltaa
kysyä apua, jos hänellä on ongelmia. Kun tiimi koostuu jäsenistä,
jotka kommunikoivat usein ja hyvin, se vaikuttaa positiivisesti
ilmapiiriin ja työn laatuun. Esimerkiksi kommunikoinnin myötä
tieto leviää tiimin jäsenten kesken, joka kasvattaa kaikkien osaamista.

\item {\bf Älykäs}

Ohjelmistokehitys on pohjimmiltaan teknistä työtä, joten
ihmistaitojen lisäksi myös älykkyys on erittäin tärkeä piirre.
Älykkyydellä tarkoitetaan tässä tapauksessa ohjelmistokehittäjää, jolla on hyvä tekninen osaaminen ja hän on nopea
ajatuksissaan. Henkilön ongelmanratkaisutaidot ovat erinomaiset
ja hän kykenee ajattelemaan asioita abstraktilla tasolla sekä ottaa
vastuuta työstään.

\item {\bf Mukava}

Mukava ohjelmistokehittäjä on sosiaalinen ja hänen kanssaan on
helppo työskennellä. Hänellä on huumorintajua ja sopivaa
tilannetajua eli esimerkiksi pystyy huomauttamaan virheistä ilman,
että toinen pahastuu. Lyhyesti sanottuna kyseisen henkilön kanssa
on hauska työskennellä.

Muita mainittuja piirteitä olivat muun muassa, että ohjelmistokehittäjä
on tietäväinen, innovatiivinen, itsenäinen, kykenee keskittymään työhönsä, ja noudattaa käytettyä prosessia.

\end{itemize}

\subsection{Persoonallisuustyypit eri kehityksen vaiheissa}

Ohjelmiston kehitys on monivaiheinen prosessi ja usein se on
jaettu vaatimusmäärittelyyn, suunnitteluun, toteutukseen, testaukseen
ja ylläpitoon. Nämä eri vaiheet vaativat erilaisia teknisiä taitoja,
jotta saadaan paras mahdollinen lopputulos. Kuitenkin teknisen
osaamisen lisäksi myös persoonallisuustyyppi vaikuttaa siihen kuinka
hyvin henkilö suoriutuu tietyistä tehtävistä tai kuinka hyvin hän
soveltuu niihin~\cite{Capretz:2003:PTS:766407.766410,Capretz:2010:MSS:1726559.1726574}. Kun luonteiltaan oikeanlaiset henkilöt valitaan suorittamaan
näitä tiettyjä kehityksen vaiheita, tuloksena on parempi lopputulos
projektin kannalta.\\

Seuraavaksi käydään edellä mainitut kehityksen vaiheet läpi yksi kerrallaan
ja tarkastellaan millaiset ohjelmistokehittäjät sopivat
parhaiten mihinkin vaiheeseen. Oletetaan, että kehittäjillä on
hyvä tekninen osaaminen kaikissa vaiheissa, jolloin tarkastelu
voidaan rajoittaa ainoastaan siihen kuinka hyvin he persoonallisuustyypeiltään soveltuvat suorittamaan tiettyjä
kehityksen vaiheita.

\subsubsection{Myers-Briggsin tyyppi-indikaattori}

Persoonallisuustyyppejä määriteltäessä käytetään apuna
Myers-Briggsin tyyppi-indikaattoria, joka jakaa ihmisen persoonallisuuden neljään eri osioon: sosiaaliseen vuorovaikutukseen, tiedonkeruuseen, päätöksentekoon ja elämän\-tyyliin~\cite{Capretz:2010:MSS:1726559.1726574,DaCunha:2007:PMA:1230819.1241672}. Joka osiossa on
kaksi eri luonteenpiirrettä.

\begin{enumerate}

\item Sosiaalinen vuorovaikutus: Ekstrovertti (E) --- Introvertti (I)

Ekstrovertit ovat sosiaalisia, ulospäinsuuntautuneita ja nauttivat
muiden ihmisten seurasta. Introvertit sen sijaan ovat sisäänpäinkääntyneitä, hiljaisia, varautuneita ja ovat mieluummin omissa oloissaan.

\item Tiedonkeruu: Tosiasiallinen (S) --- Intuitiivinen (N)

Tosiasialliset ihmiset etsivät yksityiskohtaista tietoa ja tunnettuja
tosiasioita. He uskovat enemmän konkreettisiin asioihin, jotka
voivat itse omin aistein todistaa. Intuitiiviset etsivät asioille
yhteyksiä teoreettisemman ja abstraktisemmankin tiedon pohjalta ja
he miettivät eri asioista aiheutuvia mahdollisuuksia.

\item Päätöksenteko: Ajatteleva (T) --- Tunteva (F)

Ajattelevat perustavat päätöksensä miettimällä tarkasti päätöksen syitä, seurauksia ja loogisuutta. He siis perustavat päätöksensä puhtaaseen järjen käyttöön. Tuntevilla on taipumusta tehdä päätös
henkilökohtaisten arvojen perusteella ja sen mukaan miten päätös
vaikuttaa muihin ihmisiin. Päätöksenteko-ulottuvuus vaikuttaa siihen minkälaisista tehtävistä
henkilö kiinnostuu ja kuinka tyytyväinen hän on niihin.

\item Elämäntyyli: Järjestelmällinen (J) --- Spontaani (P)

Järjestelmälliset ihmiset ovat sananmukaisesti järjestelmällisiä
ja täsmällisiä. He pitävät aikamääreistä kiinni ja suunnittelevat
asioita etukäteen. Spontaanit taas ovat joustavia ja elävät
hetkessä välittämättä suunnitelmallisuudesta ja järjestelmällisyydestä.
Elämäntyyli-ulottuvuus vaikuttaa henkilön työskentelytapoihin.

\end{enumerate}

\subsubsection{Vaatimusmäärittely}
\subsubsection{Suunnittelu}
\subsubsection{Toteutus}
\subsubsection{Testaus}
\subsubsection{Ylläpito}

\section{Ryhmätyötaitojen parantaminen}

\section{Yhteenveto}


% --- Back matter ---
%
% bibtex is used to generate the bibliography. The babplain style
% will generate numeric references (e.g. [1]) appropriate for theoretical
% computer science. If you need alphanumeric references (e.g [Tur90]), use
%
% \bibliographystyle{babalpha}
%
% instead.

\bibliographystyle{babplain}
\bibliography{../lahteet}


\end{document}